%
% Ejemplo con el formato para un trabajo de título
% de la Escuela de Ingeniería de la Universidad de Chile
%
% Fecha: 30 de Mayo de 2003
%
\documentclass[12pt,oneside,letterpaper]{report}
\usepackage[spanish]{babel}
\usepackage[ansinew]{inputenc}
\usepackage[dvips]{graphicx}
\usepackage[right=2cm,left=3cm,top=2cm,bottom=2cm,headsep=0cm,footskip=0.5cm]{geometry}

%-------------------------------------------------------------------------
%definición de comandos

\newcommand{\dotrule}[1]{\parbox[t]{#1}{\dotfill}}

%para código fuente
\newenvironment{codigoenv}
{\fontsize{10pt}{12pt} \linespread{1}} { \normalsize}

%para las imagenes
\newenvironment{figuraenv}
{\begin{figure}[htb]\begin{center}} {\end{center}\end{figure}}


\newcommand{\df}[2]{\textit{#1 (#2)}} %definicion de termino y sigla
\newcommand{\cls}[1]{\mbox{\textit{#1}}} %nombre de clase o paquete o código
\newcommand{\trm}[1]{\textit{#1}} %termino tecnico o en ingles
\newcommand{\cod}[1]{\texttt{\footnotesize #1}} %codigo fuente

%-------------------------------------------------------------------------


\linespread{1}
\setlength{\parskip}{1\baselineskip}
\parindent 1cm
\sloppy


%-------------------------------------------------------------------------

\begin{document}

%-------------------------------------------------------------------------
\thispagestyle{empty}
\begin{center}

UNIVERSIDAD DE CHILE\\
FACULTAD DE CIENCIAS FÍSICAS Y MATEMÁTICAS\\
DEPARTAMENTO DE CIENCIAS DE LA COMPUTACIÓN\\

\vspace{5cm}

INVESTIGACIÓN DE LA PLATAFORMA J2EE Y SU APLICACIÓN PRÁCTICA

\vspace{5cm}

JUAN MANUEL BARRIOS NÚÑEZ

\vfill

2003
\end{center}

%-------------------------------------------------------------------------
\newpage
\thispagestyle{empty}

\begin{center}
UNIVERSIDAD DE CHILE\\
FACULTAD DE CIENCIAS FÍSICAS Y MATEMÁTICAS\\
DEPARTAMENTO DE CIENCIAS DE LA COMPUTACIÓN

\vskip 1cm

INVESTIGACIÓN DE LA PLATAFORMA J2EE Y SU APLICACIÓN PRÁCTICA

\vskip 1cm

JUAN MANUEL BARRIOS NÚÑEZ

\vskip 1.8cm

\end{center}

\begin{flushleft}
\begin{tabular}{llccc}
COMISIÓN EXAMINADORA & & \multicolumn{3}{l}{\hspace{1.8cm} CALIFICACIONES} \\
& & \footnotesize{NOTA (n$^o$)} & \footnotesize{(Letras)} & \footnotesize{FIRMA} \\[.8cm]

PROFESOR GUÍA \\
SR. DIONISIO GONZÁLEZ & : & \dotrule{1.4cm} & \dotrule{2.8cm} & \dotrule{2.5cm} \\[.7cm]

PROFESOR CO-GUÍA \\
SR. PATRICIO INOSTROZA & : &  \dotrule{1.4cm} & \dotrule{2.8cm} & \dotrule{2.5cm} \\[.7cm]

PROFESOR INTEGRANTE  \\
SR. EDUARDO GODOY & : & \dotrule{1.4cm} & \dotrule{2.8cm} & \dotrule{2.5cm} \\[.7cm]

\\
NOTA FINAL EXAMEN DE TÍTULO & : &  \dotrule{1.4cm} & \dotrule{2.8cm} & \dotrule{2.5cm} \\

\end{tabular}
\end{flushleft}

\vskip .5cm

\begin{center}
MEMORIA PARA OPTAR AL TÍTULO DE\\
INGENIERO CIVIL EN COMPUTACIÓN

\vfill

SANTIAGO DE CHILE\\
ENERO 2003

\end{center}


%-------------------------------------------------------------------------
\newpage
\thispagestyle{empty}

\begin{flushright}
\small
\begin{tabular}{l}
RESUMEN DE LA MEMORIA\\
PARA OPTAR AL TÍTULO DE\\
INGENIERO CIVIL EN COMPUTACIÓN\\
POR: JUAN MANUEL BARRIOS NÚÑEZ\\
FECHA: 30/05/2003\\
PROF. GUÍA: SR. DIONISIO GONZÁLEZ\\
\end{tabular}
\end{flushright}

\vskip .5cm

\begin{center}
INVESTIGACIÓN DE LA PLATAFORMA J2EE Y SU APLICACIÓN PRÁCTICA
\end{center}

\vskip 1cm

%objetivos y justificación

El presente trabajo tiene como objetivo adquirir conocimientos y
experiencia teórica y práctica en el desarrollo de aplicaciones
empresariales utilizando el modelo ``Java 2 Platform, Enterprise Edition''
(J2EE). Este nuevo modelo ha tomado importancia por proponer una
arquitectura para desarrollar e integrar sistemas de una empresa,
definiendo un servidor de aplicaciones que consta de múltiples componentes
y servicios. Efectuar un estudio concreto sobre sus capacidades y elaborar
metodologías de utilización es un paso necesario que permite su aplicación
correcta en proyectos reales.

%metodología

Para conseguir este objetivo, el trabajo fue dividido en una fase de
investigación y en una fase de aplicación. En la fase de investigación se
estudió la plataforma J2EE, sus tecnologías relacionadas y los patrones de
diseño existentes para el desarrollo. En la fase de aplicación se
utilizaron los conocimientos adquiridos para el desarrollo de un proyecto
con el objetivo de encontrar una metodología de desarrollo para
aplicaciones J2EE, obtener experiencia sobre las capacidades de esta
plataforma y contar con un caso de estudio que permita apoyar el diseño y
construcción de nuevos sistemas.

%hipótesis, supuestos y resultados obtenidos

El resultado final es un informe que reúne los conocimientos necesarios
para el entendimiento de la plataforma J2EE, su servidor de aplicaciones y
sus componentes, junto con la implementación de un sistema de registro de
actividades como proyecto práctico. Con este proyecto se obtuvo una
metodología para abordar el desarrollo de un sistema J2EE, cuatro patrones
de diseño para solucionar problemas concretos en la implementación de un
sistema, y un conjunto de evaluaciones y conclusiones sobre el uso y las
capacidades de esta tecnología.

%conclusiones y recomendaciones

J2EE es una arquitectura que ha evolucionado rápidamente, para
transformarse en una opción a ser considerada para efectuar el desarrollo
de aplicaciones empresariales, sin embargo su utilización se ha visto
retrasada por la falta de conocimientos reales en su desarrollo e
implementación. Por esta razón se necesita generar conocimientos concretos
que permitan apoyar su uso correcto en aplicaciones empresariales reales,
crear nuevos casos de estudio y desarrollar nuevos patrones de diseño que
aporten con experiencia práctica en su utilización.


%-------------------------------------------------------------------------
\newpage \pagenumbering{roman}
\tableofcontents

%-------------------------------------------------------------------------


%-------------------------------------------------------------------------
\newpage
\pagenumbering{arabic}
\chapter{Presentación}

\section{Introducción}

Internet y sus servicios, particularmente la Web, tienen una gran
importancia en el desarrollo de las empresas en la actualidad, siendo
factores esenciales para la llamada \trm{Nueva Economía}.

%-------------------------------------------------------------------------
\newpage
\chapter{Componentes de J2EE}


%-------------------------------------------------------------------------
\section{JavaServer Pages}

\subsection{Sintaxis}

\subsubsection{Programación}

\begin{enumerate}
\item
Declaración. Su sintaxis es \mbox{\cod{<\%! ... \%>}}. Son utilizados para
declarar variables y métodos para la página. El cuerpo de este elemento es
traducido como instrucciones globales a la clase \cls{HttpJspPage}. Por
ejemplo:

\cod{<\%! int n; \%>}, declara una variable global a la página.

\item
Scriptlets. Su sintaxis es \cod{<\% ... \%>}. Pueden contener cualquier
fragmento de código del lenguaje de programación de la página que haya
sido declarado en la directiva \cls{page}. Los scriptlets se ejecutan en
cada proceso de \trm{request} y pueden modificar objetos, declarar
variables, llamar funciones o cualquier otra acción que proporcione el
lenguaje. El cuerpo de este elemento es copiado al método
\cls{\_jspService}.

\item
Expresiones. Su sintaxis es \cod{<\%= ... \%>}. Corresponde a una
expresión del lenguaje de programación cuyo resultado es evaluado como un
string y agregado a la respuesta. El cuerpo de este elemento es copiado al
método \cls{\_jspService} dentro de una instrucción \cls{out.write()}. Por
ejemplo:

\cod{N es un número <\%=
N>=0?\cod{"}positivo\cod{"}:\cod{"}negativo\cod{"}\%>}, escribe si N es
positivo o negativo.

\end{enumerate}

%-------------------------------------------------------------------------
\section{Enterprise JavaBeans}
\label{sec:ejb}

\subsection{Session Beans}

\subsubsection{Implementación de session beans}

La implementación de un session bean stateless o stateful está compuesta
de al menos tres archivos diferentes: dos interfaces y una clase.

\begin{codigoenv}
\begin{verbatim}
public interface Saludo extends javax.ejb.EJBObject {
  public String hola() throws java.rmi.RemoteException;
}
\end{verbatim}
\end{codigoenv}


El diagrama de las clases e interfaces involucradas en el ejemplo se puede
ver en la figura \ref{fig:simp}. Las clases e interfaces han sido
divididas según quien las proporciona, en primer lugar se encuentran las
interfaces de J2SE, luego las de J2EE y utilizando éstas se encuentran las
creadas en el ejemplo. Las interfaces ahí definidas son implementadas por
clases generadas automáticamente por el container que serán los objetos
que obtendrá y ejecutará el cliente.

\begin{figuraenv}
%comentado para que compile sin la imagen
%\includegraphics[scale=.7]{simp.eps}
\caption{Diagrama de clases para session beans} \label{fig:simp}
\end{figuraenv}



%-------------------------------------------------------------------------
\bibliographystyle{plain}
\bibliography{ejemploBibtex}

\nocite{ejb} \nocite{j2ee} \nocite{servlet} \nocite{jsp} \nocite{jndi}
\nocite{connector} \nocite{jta} \nocite{jts} \nocite{jaxp}

%\nocite{mastering} \nocite{ejbpatterns} \nocite{corepatterns}
%\nocite{corejsp} \nocite{designing}

%\nocite{jboss} \nocite{jbosscmp}

%\nocite{tutorial}

%\nocite{uml} \nocite{ensuring} \nocite{jdbc}


%-------------------------------------------------------------------------
\appendix
\newpage
\chapter{Códigos fuentes}

En este apéndice se presentarán códigos fuentes mostrando la
implementación de los patrones de diseño creados para la capa de EJB.


%-------------------------------------------------------------------------
\end{document}
