\chapter*{RESUMEN}
%................................
%................................
Este documento es un ejemplo preparado para ilustrar el uso de \AmS-\LaTeX{} versi\'on~2.2 y la clase de documento \,\verb+puctesis+\, ({\em documentclass}) para \LaTeX.

Los autores deben usar la c\'odigo en el archivo \,\verb+puctesis_ejemplo.tex+\, como modelo.  Este archivo fue utilizado para preparar este ejemplo.

El documento plantilla \,\verb+puctesis_plantilla.tex+\, ser\'a de mucha ayuda para empezar con la escritura de una nueva tesis.  Un listado {\em verbatim} del c\'odigo del documento plantilla se presenta en el Ap\'endice~\ref{ap:puctesis} de este documento.

El archivo de estilo, \,\verb+puctesis.sty,+\, y la clase de documento \,\verb+puctesis.cls,+\, est\'an basados en archivos de estilo de la {\sl American Mathematical Society\/} (AMS).  Los documentos nuevos deber\'ian emplear la clase de documento \,\verb+puctesis.cls,+\, y ser compilados usando \LaTeX{} $2_\varepsilon$.  La versi\'on actual del archivo de estilo \,\verb+puctesis.sty,+\, no debe ser utilizado, ya que no cumple con las instrucciones para la preparaci\'on de tesis.  Este archivo se provee solamente como una referencia y para permitir la compilaci\'on en la improbable situaci\'on que \LaTeX{} $2_\varepsilon$ no est\'e disponible, pero si su versi\'on anterior ({\em plain} \LaTeX).  La mayor parte del texto de este ejemplo es la misma que usualmente es provista por la AMS en sus paquetes de estilo para art\'iculos y monograf\'ias.


           %%%%%%%%%%%%%%%%%%%%%%%%%%%%%%%%%%%%%%%%%%%%%%%%%%%%
           %   PALABRAS CLAVES                                %
           %--------------------------------------------------%
           %      al final de la pagina de resumen            %
           %%%%%%%%%%%%%%%%%%%%%%%%%%%%%%%%%%%%%%%%%%%%%%%%%%%%

~\vfill
{\bf Palabras Claves:} \parbox[t]{.75\textwidth}{plantilla de tesis,
  escritura de documentos, ecuaciones diferenciales, aerodin\'amica,
  teor\'ia electromagn\'etica de ondas, an\'alisis de impactos, elasticidad,
  simulaci\'on por computador, mec\'anica cu\'antica, 
  f\'ormula de \mbox{Campbell-Baker-Hausdorff}.
}